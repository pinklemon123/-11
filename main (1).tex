
\documentclass{article}
\usepackage{amsmath}
\usepackage{amssymb}
\usepackage[utf8]{inputenc}
\usepackage{geometry}
\geometry{a4paper, margin=1in}
\title{JobGraph Basic definition}
\author{}
\date{}

\begin{document}

\maketitle

\section*{1. Basic Definitions}

Let $G = (V, E)$ be a \textbf{directed graph}, where:

\begin{itemize}
    \item $V$ is the set of vertices. Each vertex $v \in V$ has a unique identifier $id(v) \in S$, where $S$ is a set of strings.
    \item $E \subseteq V \times V$ is the set of edges, representing dependency relationships between tasks.
\end{itemize}

\bigskip

\noindent
The graph $G$ maintained by the \texttt{JobGraph} class satisfies:

\[
G = (V, E), \quad \text{where } V \subseteq S, \quad E \subseteq V \times V
\]

\section*{2. Formal Description of Methods}

\subsection*{(1) Add Vertex (\texttt{add\_vertex})}

Given a vertex $v$ (identified by \texttt{id}), update the graph as follows:

\[
V' = V \cup \{v\}, \quad E' = E
\]

That is:

\[
G' = (V \cup \{v\}, E)
\]

\subsection*{(2) Get Vertex Data (\texttt{vertices\_data})}

Return all vertices and their (currently empty) attributes:

\[
\texttt{vertices\_data}(G) = \{ (v, \emptyset) \mid v \in V \}
\]



\section*{3. Basic Definitions}

Let $G = (V, E)$ be a \textbf{directed graph}, where:

\begin{itemize}
    \item $V$ is the set of vertices. Each vertex $v \in V$ has a unique identifier $id(v) \in S$, where $S$ is a set of strings.
    \item $E \subseteq V \times V$ is the set of edges, representing dependencies between tasks.
\end{itemize}

The graph $G$ maintained by the \texttt{JobGraph} class satisfies:

\[
G = (V, E), \quad \text{where } V \subseteq S,\quad E \subseteq V \times V
\]

\section*{4. Formal Description of Methods}

\subsection*{(1) Add Vertex (\texttt{add\_vertex})}

Given a vertex $v$ (identified by \texttt{id}), update the graph as follows:

\[
V' = V \cup \{v\}, \quad E' = E
\]

That is:
\[
G' = (V \cup \{v\}, E)
\]

\subsection*{(2) Get Vertex Data (\texttt{vertices\_data})}

Return all vertices along with their (currently empty) attributes:

\[
\texttt{vertices\_data}(G) = \{ (v, \emptyset) \mid v \in V \}
\]

\subsection*{(3) Merge Graphs (\texttt{update})}

Given another graph $G_2 = (V_2, E_2)$, update the current graph as:

\[
V' = V \cup V_2,\quad E' = E \cup E_2
\]

That is:
\[
G' = (V \cup V_2, E \cup E_2)
\]

\subsection*{(4) Remove Vertex (\texttt{remove\_vertex})}

Given a vertex $v$, update the graph as:

\[
V' = V \setminus \{v\}
\]
\[
E' = E \setminus \{(u, w) \mid u = v \lor w = v \}
\]

That is, remove $v$ and all edges connected to it.

\subsection*{(5) Extract Subgraph (\texttt{subgraph})}

Given a subset of vertices $V_s \subseteq V$, return the subgraph:

\[
G_s = (V_s, E_s), \quad \text{where } E_s = E \cap (V_s \times V_s)
\]

\section*{5. Invariants}

The graph $G$ maintained by the \texttt{JobGraph} class must always satisfy the following invariants:

\begin{itemize}
    \item \textbf{Unique Identifiers}: $\forall v_1, v_2 \in V,\; id(v_1) = id(v_2) \Rightarrow v_1 = v_2$ \\
    (Each vertex has a unique ID.)
    
    \item \textbf{No Self-Loops (unless explicitly allowed)}: $\forall v \in V,\; (v, v) \notin E$ \\
    (By default, no vertex has a self-dependency.)

    \item \textbf{Acyclic (for task scheduling)}: \\
    The dependency graph must be acyclic if used for task scheduling, i.e., $G$ must be a Directed Acyclic Graph (DAG).
\end{itemize}

\section*{6. Summary}

The mathematical essence of a \texttt{JobGraph} is a \textbf{directed graph} $G = (V, E)$, and its methods correspond to standard set operations:

\begin{itemize}
    \item \textbf{Vertex operations}: union, difference, and subset operations on $V$.
    \item \textbf{Edge operations}: union, difference, and restriction operations on $E$.
\end{itemize}


\end{document}
